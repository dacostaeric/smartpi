% !TEX TS-program = pdflatex
% !TEX encoding = UTF-8 Unicode

\documentclass[11pt,a4paper]{article}

\usepackage[utf8]{inputenc}

\usepackage[T1]{fontenc}
\usepackage[german]{babel}
\selectlanguage{german}

\usepackage{graphicx}

\pagestyle{plain}

\title{SmartPi\\ \begin{large} Computer Architecture and Operating Systems\\ Universität Basel\end{large}}
\author{Eric da Costa, Alessandro Pittori}
\date{27. Januar 2019}

\begin{document}
\maketitle

%\tableofcontents
\section{Inspiration}
automatisierung des heims, neubauten teilweise schon integriert
smarthome integrierte produkte (glühbirnen, smart assistants) -> idee bekommen, was so dahintersteckt
Die Idee eines Webserver basierten Smart Homes kam uns erstmals, als die Beispielprojekte vorgestellt wurden. Unter diesen war eine Wetterstation, auf die man über einen Webserver zugreifen konnte. Wir hatten beide schon zuvor Erfahrung mit Webdesign und Alessandro zudem /SYN/Erfahrung mit Webhosting, dem Raspberry Pi und der Kombination der beiden. Des weiteren hatte er einen Online Kurs des Hasso Platter Instituts /SRC/openhpi
\cite{smartmirror}

- am morgen aufstehen -> up to date mit termine email etc
- ohne handy

- hackerhouse smart mirror
- webserver, "remote" access im netzwerk
- tts, smart assistant-like

questions to examine
verbinden verschiedener technologien
java <-> react
java <-> python
zentrale steuereinheit mit schnittstelle (server) im netzwerk verfügbar


\section{Technologien}
- rpi, schon zuvor verwendet mit webserver
- java, kennen uns am besten aus
- python, auslesen der sensoren
- javascript
	moderne herangehensweise an webdesign und server
	- react für frontend
		- npm
		- fetch api um daten zu beschaffen
	- nodejs server für backend
- http
- google api
- java mail

\subsection{APIs}
- openweathermap
- goolgle calendar
- javax.mail imaps, smtp
- marytts
\subsection{Hardware}
- 7in lcd von smart home kit
- sensoren von smart home kit
- lautsprecher von früher
- rpi 2b
- sperrholz + acryl

\section{Aufteilung}
\subsection{Alessandro}
- react
- webserver
- wetter
- gui/ux
- icons sophia

\subsection{Eric}
- java
- python
- email
- wecker
- kalender
- sensoren
- tts

\section{Implementierung}
\subsection{Architektur} % ale
- node server
- react front end
- makeshift database using json files
- java und node server schreiben json
- java im hintergrund für komplexere protokolle (email, calendar api)
- python sensoren auslesen

\subsection{Wecker}

\subsection{E-Mail}

\subsection{Kalender}

\subsection{Sensoren}

\subsection{React}

\subsection{Webserver}

\subsection{Gerhäuse und Zusammenbau}
- blender sketch

\section{Abschliessend}
\subsection{Resultat}
\subsection{Fazit}

\bibliographystyle{plain}
\bibliography{CAOS-Project-Report_SmartPi_Eric_da_Costa_Alessandro_Pittori}

\end{document}
