% !TEX TS-program = pdflatex
% !TEX encoding = UTF-8 Unicode

\documentclass[11pt,a4paper]{article}

\usepackage[utf8]{inputenc}

\usepackage[T1]{fontenc}
\usepackage[german]{babel}
\selectlanguage{german}

\usepackage{graphicx}

\pagestyle{plain}

\title{SmartPi\\ \begin{large} Computer Architecture and Operating Systems\\ Universität Basel\end{large}}
\author{Eric da Costa, Alessandro Pittori}
\date{27. Januar 2019}

\begin{document}
\maketitle

%\tableofcontents
\section{Inspiration}
automatisierung des heims, neubauten teilweise schon integriert
smarthome integrierte produkte (glühbirnen, smart assistants) -> idee bekommen, was so dahintersteckt
Die Idee eines Webserver basierten Smart Homes kam uns erstmals, als die Beispielprojekte vorgestellt wurden. Unter diesen war eine Wetterstation, auf die man über einen Webserver zugreifen konnte. Wir hatten beide schon zuvor Erfahrung mit Webdesign und Alessandro zudem /SYN/Erfahrung mit Webhosting, dem Raspberry Pi und der Kombination der beiden. Des weiteren hatte er einen Online Kurs des Hasso Platter Instituts /SRC/openhpi
\cite{smartmirror}

- am morgen aufstehen -> up to date mit termine email etc
- ohne handy

- hackerhouse smart mirror
- webserver, "remote" access im netzwerk
- tts, smart assistant-like

questions to examine
verbinden verschiedener technologien
java <-> react
java <-> python
zentrale steuereinheit mit schnittstelle (server) im netzwerk verfügbar


\section{Technologien}
- rpi, schon zuvor verwendet mit webserver
- java, kennen uns am besten aus
- python, auslesen der sensoren
- javascript
	moderne herangehensweise an webdesign und server
	- react für frontend
		- npm
		- fetch api um daten zu beschaffen
	- nodejs server für backend
- http
- google api
- java mail

\subsection{APIs}
- openweathermap
- goolgle calendar
- javax.mail imaps, smtp
- marytts
\subsection{Hardware}
- 7in lcd von smart home kit
- sensoren von smart home kit
- lautsprecher von früher
- rpi 2b
- sperrholz + acryl

\section{Aufteilung}
\subsection{Alessandro}
- react
- webserver
- wetter
- gui/ux
- icons sophia

\subsection{Eric}
Eric hat bei dem Projekt die Java Programme für das Empfangen der E-Mails und der Kalenderdaten geschrieben. Die Interaktion mit den Sensoren, welche in Python programmiert wurde und die Kommunikation zwischen Python und Java wurden auch von ihm \"ubernommen. Zusätzlich hat er sich auch um die Text-to-Speech Funtkionalit\"at vom Projekt gek\"ummert. Die Daten wurden dann an React weitergeschickt, um dort verarbeitet zu werden. 
 
-java -
- python  -
- email - 
- wecker ALE
- kalender -
- sensoren -
- tts -

\section{Implementierung}
\subsection{Architektur} % ale
- node server
- react front end
- makeshift database using json files
- java und node server schreiben json
- java im hintergrund für komplexere protokolle (email, calendar api)
- python sensoren auslesen

\subsection{Wecker}

\subsection{E-Mail}
Das Ziel bei der E-Mail war, dass man nur Nachrichten empfangen kann, da f\"ur das Projekt nicht geplant war auf E-Mails zu antworten. Wie bereits erw\"ahnt haben wir uns f\"ur die Java Mail API entschieden, da wir eine M\"oglichkeit haben wollten, jegliche Adressen zu verwenden und uns nicht auf eine, wie z.B. Gmail, zu beschr\"anken.

\subsection{Kalender}
Im Gegensatz zur Verwaltung der E-Mails haben wir uns hier beschlossen \textit{Google Calendar} zu verwenden. Diese Wahl haben wir getroffen, weil wir beide privat selbst \textit{Google Calendar} ben\"utzen.

\subsection{Sensoren}
Die verwendeten Sensoren wurden mit einem Python Skript ausgelesen und entsprechend ausgewertet. Die Daten werden bei jedem Aufruf des Skripts \textit{geprintet}.

\subsection{Java}
Die Kalender Events und die empfangenen Mails werden anschliessend als \textit{ArrayList<Map<String, String>>} weitergegeben \textbf{WEM?} um dort weiterverwendet zu werden. Die Sensordaten werden als \textit{Map<String, Object>} weitergegeben.

\subsection{Text-to-Speech}
Nach dem der Wecker vom Benutzer ausgeschaltet wurde, wird von der Text-to-Speech Funktionalit\"at zuersts das n\"achste Event vorgelesen und anschliessend alle neuen Nachrichten.
\subsection{React}

\subsection{Webserver}

\subsection{Gerhäuse und Zusammenbau}
- blender sketch

\section{Abschliessend}
\subsection{Resultat}
\subsection{Fazit}

\bibliographystyle{plain}
\bibliography{CAOS-Project-Report_SmartPi_Eric_da_Costa_Alessandro_Pittori}

\end{document}
